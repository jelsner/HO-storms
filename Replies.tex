% Options for packages loaded elsewhere
\PassOptionsToPackage{unicode}{hyperref}
\PassOptionsToPackage{hyphens}{url}
%
\documentclass[
]{article}
\usepackage{amsmath,amssymb}
\usepackage{iftex}
\ifPDFTeX
  \usepackage[T1]{fontenc}
  \usepackage[utf8]{inputenc}
  \usepackage{textcomp} % provide euro and other symbols
\else % if luatex or xetex
  \usepackage{unicode-math} % this also loads fontspec
  \defaultfontfeatures{Scale=MatchLowercase}
  \defaultfontfeatures[\rmfamily]{Ligatures=TeX,Scale=1}
\fi
\usepackage{lmodern}
\ifPDFTeX\else
  % xetex/luatex font selection
\fi
% Use upquote if available, for straight quotes in verbatim environments
\IfFileExists{upquote.sty}{\usepackage{upquote}}{}
\IfFileExists{microtype.sty}{% use microtype if available
  \usepackage[]{microtype}
  \UseMicrotypeSet[protrusion]{basicmath} % disable protrusion for tt fonts
}{}
\makeatletter
\@ifundefined{KOMAClassName}{% if non-KOMA class
  \IfFileExists{parskip.sty}{%
    \usepackage{parskip}
  }{% else
    \setlength{\parindent}{0pt}
    \setlength{\parskip}{6pt plus 2pt minus 1pt}}
}{% if KOMA class
  \KOMAoptions{parskip=half}}
\makeatother
\usepackage{xcolor}
\usepackage[margin=1in]{geometry}
\usepackage{color}
\usepackage{fancyvrb}
\newcommand{\VerbBar}{|}
\newcommand{\VERB}{\Verb[commandchars=\\\{\}]}
\DefineVerbatimEnvironment{Highlighting}{Verbatim}{commandchars=\\\{\}}
% Add ',fontsize=\small' for more characters per line
\usepackage{framed}
\definecolor{shadecolor}{RGB}{248,248,248}
\newenvironment{Shaded}{\begin{snugshade}}{\end{snugshade}}
\newcommand{\AlertTok}[1]{\textcolor[rgb]{0.94,0.16,0.16}{#1}}
\newcommand{\AnnotationTok}[1]{\textcolor[rgb]{0.56,0.35,0.01}{\textbf{\textit{#1}}}}
\newcommand{\AttributeTok}[1]{\textcolor[rgb]{0.13,0.29,0.53}{#1}}
\newcommand{\BaseNTok}[1]{\textcolor[rgb]{0.00,0.00,0.81}{#1}}
\newcommand{\BuiltInTok}[1]{#1}
\newcommand{\CharTok}[1]{\textcolor[rgb]{0.31,0.60,0.02}{#1}}
\newcommand{\CommentTok}[1]{\textcolor[rgb]{0.56,0.35,0.01}{\textit{#1}}}
\newcommand{\CommentVarTok}[1]{\textcolor[rgb]{0.56,0.35,0.01}{\textbf{\textit{#1}}}}
\newcommand{\ConstantTok}[1]{\textcolor[rgb]{0.56,0.35,0.01}{#1}}
\newcommand{\ControlFlowTok}[1]{\textcolor[rgb]{0.13,0.29,0.53}{\textbf{#1}}}
\newcommand{\DataTypeTok}[1]{\textcolor[rgb]{0.13,0.29,0.53}{#1}}
\newcommand{\DecValTok}[1]{\textcolor[rgb]{0.00,0.00,0.81}{#1}}
\newcommand{\DocumentationTok}[1]{\textcolor[rgb]{0.56,0.35,0.01}{\textbf{\textit{#1}}}}
\newcommand{\ErrorTok}[1]{\textcolor[rgb]{0.64,0.00,0.00}{\textbf{#1}}}
\newcommand{\ExtensionTok}[1]{#1}
\newcommand{\FloatTok}[1]{\textcolor[rgb]{0.00,0.00,0.81}{#1}}
\newcommand{\FunctionTok}[1]{\textcolor[rgb]{0.13,0.29,0.53}{\textbf{#1}}}
\newcommand{\ImportTok}[1]{#1}
\newcommand{\InformationTok}[1]{\textcolor[rgb]{0.56,0.35,0.01}{\textbf{\textit{#1}}}}
\newcommand{\KeywordTok}[1]{\textcolor[rgb]{0.13,0.29,0.53}{\textbf{#1}}}
\newcommand{\NormalTok}[1]{#1}
\newcommand{\OperatorTok}[1]{\textcolor[rgb]{0.81,0.36,0.00}{\textbf{#1}}}
\newcommand{\OtherTok}[1]{\textcolor[rgb]{0.56,0.35,0.01}{#1}}
\newcommand{\PreprocessorTok}[1]{\textcolor[rgb]{0.56,0.35,0.01}{\textit{#1}}}
\newcommand{\RegionMarkerTok}[1]{#1}
\newcommand{\SpecialCharTok}[1]{\textcolor[rgb]{0.81,0.36,0.00}{\textbf{#1}}}
\newcommand{\SpecialStringTok}[1]{\textcolor[rgb]{0.31,0.60,0.02}{#1}}
\newcommand{\StringTok}[1]{\textcolor[rgb]{0.31,0.60,0.02}{#1}}
\newcommand{\VariableTok}[1]{\textcolor[rgb]{0.00,0.00,0.00}{#1}}
\newcommand{\VerbatimStringTok}[1]{\textcolor[rgb]{0.31,0.60,0.02}{#1}}
\newcommand{\WarningTok}[1]{\textcolor[rgb]{0.56,0.35,0.01}{\textbf{\textit{#1}}}}
\usepackage{graphicx}
\makeatletter
\def\maxwidth{\ifdim\Gin@nat@width>\linewidth\linewidth\else\Gin@nat@width\fi}
\def\maxheight{\ifdim\Gin@nat@height>\textheight\textheight\else\Gin@nat@height\fi}
\makeatother
% Scale images if necessary, so that they will not overflow the page
% margins by default, and it is still possible to overwrite the defaults
% using explicit options in \includegraphics[width, height, ...]{}
\setkeys{Gin}{width=\maxwidth,height=\maxheight,keepaspectratio}
% Set default figure placement to htbp
\makeatletter
\def\fps@figure{htbp}
\makeatother
\setlength{\emergencystretch}{3em} % prevent overfull lines
\providecommand{\tightlist}{%
  \setlength{\itemsep}{0pt}\setlength{\parskip}{0pt}}
\setcounter{secnumdepth}{-\maxdimen} % remove section numbering
\ifLuaTeX
  \usepackage{selnolig}  % disable illegal ligatures
\fi
\usepackage{bookmark}
\IfFileExists{xurl.sty}{\usepackage{xurl}}{} % add URL line breaks if available
\urlstyle{same}
\hypersetup{
  pdftitle={Untitled},
  hidelinks,
  pdfcreator={LaTeX via pandoc}}

\title{Untitled}
\author{}
\date{\vspace{-2.5em}2025-03-26}

\begin{document}
\maketitle

\begin{Shaded}
\begin{Highlighting}[]
\NormalTok{tinytex}\SpecialCharTok{::}\FunctionTok{reinstall\_tinytex}\NormalTok{(}\AttributeTok{repository =} \StringTok{"illinois"}\NormalTok{)}
\end{Highlighting}
\end{Shaded}

\begin{verbatim}
## If reinstallation fails, try install_tinytex() again. Then install the following packages:
## 
## tinytex::tlmgr_install(c("amscls", "amsfonts", "amsmath", "atbegshi", "atveryend", "auxhook", "babel", "bibtex", "bigintcalc", "bitset", "bookmark", "booktabs", "cm", "ctablestack", "dehyph", "dvipdfmx", "dvips", "ec", "epstopdf", "epstopdf-pkg", "etex", "etexcmds", "etoolbox", "euenc", "everyshi", "extractbb", "fancyvrb", "filehook", "firstaid", "float", "fontspec", "framed", "geometry", "gettitlestring", "glyphlist", "graphics", "graphics-cfg", "graphics-def", "grffile", "helvetic", "hycolor", "hyperref", "hyph-utf8", "hyphen-base", "iftex", "inconsolata", "infwarerr", "intcalc", "knuth-lib", "kpathsea", "kvdefinekeys", "kvoptions", "kvsetkeys", "l3backend", "l3kernel", "l3packages", "latex", "latex-amsmath-dev", "latex-bin", "latex-fonts", "latex-tools-dev", "latexconfig", "latexmk", "letltxmacro", "lm", "lm-math", "ltxcmds", "lua-alt-getopt", "lua-uni-algos", "luahbtex", "lualatex-math", "lualibs", "luaotfload", "luatex", "luatexbase", "mdwtools", "metafont", "mfware", "modes", "natbib", "pdfescape", "pdftex", "pdftexcmds", "plain", "psnfss", "refcount", "rerunfilecheck", "scheme-infraonly", "selnolig", "stringenc", "symbol", "tex", "tex-ini-files", "texlive-scripts", "texlive-scripts-extra", "texlive.infra", "times", "tipa", "tlgpg", "tools", "unicode-data", "unicode-math", "uniquecounter", "url", "xcolor", "xetex", "xetexconfig", "xkeyval", "xunicode", "zapfding"))
\end{verbatim}

\begin{verbatim}
## The directory /Users/jameselsner/Library/TinyTeX/texmf-local is not empty. It will be backed up to /var/folders/vy/9httcfss517d71yym592jt5m0000gn/T//Rtmpqr7ong/file121d65c7e2c2a and restored later.
\end{verbatim}

\begin{verbatim}
## tlmgr install everyshi grffile tlgpg
\end{verbatim}

\begin{verbatim}
## tlmgr update --self
\end{verbatim}

\begin{verbatim}
## tlmgr install everyshi grffile tlgpg
\end{verbatim}

\begin{verbatim}
## tlmgr --repository http://www.preining.info/tlgpg/ install tlgpg
\end{verbatim}

\begin{verbatim}
## tlmgr option repository 'https://ctan.math.illinois.edu/systems/texlive/tlnet'
\end{verbatim}

\begin{verbatim}
## tlmgr update --list
\end{verbatim}

1. Yes, Equation (3) in the paper defines a model where the left-hand
side is expressed as cloglog(\(\pi\)), which is the complementary
log-log transformation of \(\pi\). This transformation is used because
th probability \(\pi\) is modeled as a function of covariates.

If you are obtaining values of \(r\) similar to those in Figure 3 by
considering the left-hand side of Eq (3) as \(\pi\) rather than
cloglog(\(\pi\)), it suggests a possible misinterpretation. The correct
interpretation should involve taking the inverse of the complementary
log-log function to retrieve \(\pi\) from the regression output. This
means: \[
\pi = 1 − \exp(-\exp(\alpha_0 + \alpha_1 SOI + \alpha_2 NAO)) 
\]

You should apply the inverse cloglog transformation to correctly
interpret the predicted probabilities.

2. Yes, the binomial parameter \(p\) can be calculated using the
equation: \[ 
    \lambda = r(1 + p)
    \]

where: \(\lambda\) is the expected annual hurricane frequency, \(r\) is
the expected annual cluster rate, \(p\) represents the probability that
a cluster contains two hurricanes rather than just one.

Implication for \(p\). If both \(\lambda\) and \(r\) are functions of
climate covariates (e.g., NAO and SOI), then \(p\) is also dependent on
these covariates: \[
p = \frac{\lambda}{r} - 1
\]

This equation suggests that \(p\) is not necessarily constant. Instead,
it varies with the ratio of hurricane frequency to cluster rate, which
could change depending on environmental conditions. We also mention that
by comparing the coefficients in Equations (4) and (5), a regression
model for \(p\) can be derived, further supporting the idea that \(p\)
is influenced by covariates. If you assume that \(p\) is constant, you
might be oversimplifying the clustering behavior of hurricanes. Instead,
modeling \(p\) as a function of climate variables could provide a more
accurate representation of hurricane occurrence patterns.

3. You're on the right track in considering \(P(H=0|r,p)\). In the
cluster model, the total number of hurricanes in a given year is given
by: \[
H = N + M
\] where: \(N \sim\) Poisson(\(r\)) and where \(M \sim\)
Binomial(\(N, p\)) is the number of extra hurricanes from a cluster of
size 2.

For the special case \(k=0\) (i.e., no hurricanes in a year), this means
there must be no clusters at all. That is, \(N = 0\), because if at
least one cluster existed, there would be at least one hurricane.

Since \(N\) follows a Poisson distribution, the probability of no
clusters occurring is: \[
P(N=0|r) = \frac{e^{-r} r^0}{0!} = e^{-r}
\] Thus, the probability that there are no hurricanes in a given year
is: \[
P(H=0|r,p) = P(N=0) = e^{-r}
\]

Which matches your intuition! The expression involving \(k-i\) and \(i\)
does not apply in this case because there are no terms to sum with
\(N=0\). The binomial part of the model (which governs cluster sizes)
only comes into play when at least one cluster is present.

So, in summary, your initial thought was correct: \[
P(H=0|r, p) = e^{-r}
\]

The undefined expression issue does not arise because the summation
formula is only used for \(k \gt 0\) while for \(k=0\) we directly use
the Poisson probability of zero clusters.

4. Here is the approach for generating estimated annual hurricane counts
for \(X\) years using the cluster model.

Step 1: Since the number of clusters per year, \(N\) follows a Poisson
distribution with rate \(r\) \(N \sim\) Poisson(\(r\)). For each year,
generate a random draw from this Poisson distribution to determine the
number of clusters that occur in that year.

Step 2: Each cluster contains either one or two hurricanes, governed by
a Binomial process \(M \sim\) Binomial(\(N\), \(p\)) where \(N\) is the
number of clusters from step 1 and \(M\) is the number of additional
hurricanes from clusters that contain two hurricanes (each cluster
contributes at least one hurricane, and an additional one with
probability \(p\)).

Step 3: Then the total number of hurricanes in a given year is
\(H = N + M\)

Step 4: Repeat for \(X\) number of years

\end{document}
